\section{Technologies}
The section describes the tools used in developing the application. 
As discussed in \autoref{chap:tech}, the application is mainly built using Kotlin supported by the following tools and technologies.

\subsection{Build Tool}
In this project, the default build tool for Kotlin, which is Gradle\footnote{\emph{Gradle} is a build automation tool used for  building, testing, and deploying software projects. Homepage: \url{https://gradle.org/}}, is used.
The Gradle Wrapper\footnote{\emph{Gradle Wrapper} is a script that runs Gradle commands without requiring an installation of Gradle on the local machine. URL: \url{https://docs.gradle.org/current/userguide/gradle_wrapper.html}} is naturally provided with Gradle. 
By using the script, tasks such as build, test, and run can be easily executed.

\subsection{Code Style}
This project follows the guidelines and conventions enforced by Lint\footnote{\emph{lint} is the default code style checker in Android Studio for Kotlin. URL: \url{https://developer.android.com/studio/write/lint}}.
Lint enforces coding standards and helps identify code style violations, ensuring a cleaner code. 
Additionally, the checkstyle runs automatically before each commit to ensure that the project follows to the coding conventions.

\subsection{PolarBleSDK}
\label{chap:polarblesdk}
The application uses the PolarBle Software Development Kit (SDK)\footnote{\emph{PolarBleSDK} maintains bluetooth low energy connection for sensors made by Polar. Github repository: \url{https://github.com/polarofficial/polar-ble-sdk/}} to establish and maintain connection between the heart rate sensor and the application.
Furthermore, the SDK provides tools and functionalities for data retrieval from the heart rate sensor.
Additionally, PolarBleSDK is compatible with Polar heart rate monitor devices such as H10 and Verity Sense.

\subsection{CI/CD}
In this project, Github Actions\footnote{\emph{Github Action} is a CI/CD platform available for repositories hosted in Github. Homepage: \url{https://github.com/features/actions}} is utilized as the CI/CD tool. 
The CI/CD pipeline is configured to be triggered automatically when there is a pull request to the main branch and when the branch is successfully merged. The configured CI/CD actions in this project is run test and build. 
This ensures that the code changes go through a standardized build and test process, maintaining the overall quality and stability of the project.

\subsection{Database}
In this project, Room is utilized as the database framework, which is based on SQLite.
The database schema consists of three entities: \texttt{User}, \texttt{Exercise}, and \texttt{Heartrate}, each representing a different data model.
These entities are accompanied with their own data access objects\footnote{\emph{Data Access Object (DAO)} serves as the main access point for interacting with the database. URL: \url{https://developer.android.com/training/data-storage/room/accessing-data}}: \texttt{UserDao}, \texttt{ExerciseDao}, and \texttt{HeartrateDao}, which allow interaction with the database. These DAOs define methods for querying data specific to each entities to the database.

In addition, two TypeConverters, namely DateTypeConverter and TimestampTypeConverter, are implemented to handle conversions between complex data types like Date and Timestamp to their corresponding storage representations in the database.

Room features an auto-migration functionality, where it automatically handles schema changes and data migration whenever a new version of the database is detected without the need of migration queries.
But in case of a conflict between the versions of the database, a SQL query for migration must be defined, ensuring seamless migration of the database.

\subsection{Dependency Injection}
To handle the dependency injection in this project, Hilt\footnote{\emph{Hilt} is a dependency injection library for Android. Homepage: \url{https://developer.android.com/training/dependency-injection/hilt-android}} is used. 
It enables the injection of DAOs into their respective repositories and repositories into their corresponding \emph{view models} and other components that required to interact with the database. 
In order to maintain a single instance of the database and its DAOs throughout the entire lifecycle of the application, they are marked with \texttt{@Singleton} annotation. 

\subsection{Material Design}
To help developing a visually appealing and user-friendly interface, Material Design is used to develop the user interface as it contains several components that are suitable for the project. 
For instance, SwitchMaterial, MaterialButton, BottomNavigation.

\subsection{MPAndroidChart}
Additionally, MPAndroidChart helps generating an interactive and dynamic graph for visualizing heart rate history.
In the scope of this project, it is required to implement a line graph to visualize the heart rate data. 
Therefore, the LineChart\footnote{The implementation using \emph{LineChart} is discussed in \autoref{chap:view_impl}} in the MPAndroidChart is used to implement the heart rate data visualization.

