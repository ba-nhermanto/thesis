\section{Model}
\label{chap:model_impl}
This chapter explains implementation details of the \emph{models} defined in \autoref{chap:model_design}.

\subsection{Entities and Relationship}
\label{chap:objectmodel_impl}
In this subsection, the detailed implementation of the entities within the system and their relationships with each other will be discussed.

\subsubsection{User Entity}
The \verb;User; entity is responsible for storing and managing user data within the system.
It serves as a representation of an individual user and holds necessary data mentioned in \autoref{fig:entity_diagram} to support other functionalities.
Additionally, the \verb;User; entity has functions to calculate the user's age based on their date of birth and to estimate the maximum heart rate using the formula mentioned in \autoref{chap:activity_intensity}.

\subsubsection{Exercise Entity}
The \verb;Exercise; entity manages information related to the user's exercises. 
This entity is responsible for holding a record of user's exercise, for instance, duration of the exercise, heart rate bpm records, and the calories burned during the exercise.

\subsubsection{Heartrate Entity}
The \verb;Heartrate; entity is responsible for storing heart rate information of users. 
This entity manages data that is valuable for monitoring users' cardiovascular health and exercise intensity such as the heart rate bpm.

\subsubsection{UserAndExercise Relationship}
The \verb;UserAndExercise; relationship establishes a one-to-many relationship between \verb;User; and \verb;Exercise;. 
This relationship allows \verb;User; to have multiple \verb;Exercise; associated with their profile. 
The relationship is defined with the "on delete cascade" constraint, ensuring that if a \verb;User; is deleted, all associated \verb;Exercise; are also deleted.

\subsubsection{UserAndHeartrate Relationship}
The \verb;UserAndHeartrate; relationship represents a one-to-many relationship between \verb;User; and \verb;Heartrate;. 
\verb;User; can have multiple \verb;Heartrate; entries associated with them. Similar to the The \verb;UserAndExercise; relationship, the "on delete cascade" constraint ensures that if a \verb;User; is deleted, their related \verb;Heartrate; data is also deleted.

\subsubsection{Activity Enum}
The \verb;Activity; enum is a predefined set of values that represents different levels of activity intensity. 
It provides a standardized way to categorize the intensity of physical activities based on the percentage of the user's maximum heart rate. 
The percentage intensity values mentioned in \autoref{chap:activity_intensity} are mapped to the activity enum, supporting \verb;ActivityService; to classify activity intensity level.


\subsection{Services}
This subsection provides the details of the implementation of the services within the system.

\subsubsection{Connection Service}
The implementation of the \verb;ConnectionService; follows the concept outlined in \autoref{chap:hr_monitor_design}. It focuses on establishing a connection with the heart rate sensor and listening to the heart rate data broadcasted by the heart rate sensor. 
It is implemented as a \verb;Service; class\footnote{A \emph{Service} is an application component that can perform long-running operations in the background. URL: \url{https://developer.android.com/guide/components/services}}, allowing it to run in the background even when the application is not in the foreground.

The lifecycle of the \verb;ConnectionService; is managed by \verb;ConnectionServiceManager;, which handles the start and stop operations. 
\texttt{ConnectionServiceManager} sends the active user's device id to the \texttt{ConnectionService} by attaching it to the \texttt{Intent}\footnote{\emph{Intent} is used to start, stop, and communicate with \emph{Service}. URL: \url{https://developer.android.com/reference/android/content/Intent}} used to initiate the \texttt{ConnectionService},
as the device id is required to start a connection to the heart rate sensor.
When the service starts, it automatically initiates the connection with the heart rate sensor.

To establish the connection with the heart rate sensor, the \verb;ConnectionService; utilizes the PolarBleSdk library.
Once the connection to the heart rate sensor is established, \verb;ConnectionService; listens to the broadcasted heart rate data and publishes the data to \verb;HrEventBus;.
This allows other components of the system, such as the \verb;HrViewModel;, to subscribe to the \verb;HrEventBus; and receive real-time heart rate updates. 
On the other hand, when the service is stopped by the \verb;ConnectionServiceManager;, the connection is gracefully terminated and the service will be stopped.

\subsubsection{Activity Service}
The \verb;ActivityService; is implemented according to \autoref{chap:activity_monitor_design}.
First, it listens to the \verb;HrEventBus;, which receives real-time heart rate data from the \verb;ConnectionService;. 
By subscribing to the \verb;HrEventBus;, the \verb;ActivityService; receives the latest heart rate data.
Every time heart rate data is received, this service determines the user's current activity according to \autoref{chap:activity_intensity}.
Once the service has classified the intensity of the activity, the result will be published to \verb;ActivityEventBus;, allowing other components such as \verb;HrViewModel; to subscribe and receive the results.

\subsubsection{Exercise Service}
The implementation of this service follows the design described in \autoref{chap:burnedcalories_design}.
It records exercises done by the users and calculate the energy expenditure in that exercise.
Similar to the \verb;ConnectionService;, the lifecycle of the \verb;ExerciseService; is managed by another class named \verb;ExerciseServiceManager;, which handles the start and stop operations. 
When the user starts an exercise, meaning the \texttt{ExerciseService} starts, it finds an active exercise created by \texttt{ExerciseViewModel} and the active user within the database. 

For the calculation of burned calories, \verb;ExerciseService; needs the user's heart rate, To retrieve the actual heart rate data, it actively listens to \verb;HeartrateEvent; published to \verb;HeartrateEventBus;.
Every time heart rate event is received, \verb;ExerciseService; adds the heart rate bpm into a collection of \verb;Int; and updates the active exercise. The update process includes burned calories calculation based on \autoref{chap:energy_expenditure}. 
Once the exercise is processed, it will be published to \verb;ExerciseEventBus;, which allows \verb;ExerciseViewModel; to be up-to-date with the active exercise.

When the user stops the exercise, the service is stopped by \verb;ExerciseServiceManager;, then the service unsubscribes from the \verb;HeartrateEventBus;, mark the exercise as done, process and save the exercise to the database and once it is saved, it is published to \verb;ExerciseEventBus;.
To prevent \url{ANR}\footnote{\emph{Android Not Responsive (ANR)} occurs when the main thread is blocked. URL: \url{https://developer.android.com/topic/performance/vitals/anr}}, the process to save the exercise to the database is executed on a \emph{coroutine}\footnote{A \emph{coroutine} is a concurrency design pattern that you can use on Android to simplify code that executes asynchronously. URL: \url{https://developer.android.com/kotlin/coroutines}}.
