\chapter{Conception}

This is the Conception chapter 

\section{System Architecture}
System Architecture section, show some diagrams

\section{Software Architecture}
Software Architecture section, show some diagrams

\section{Technologies}
Tech section
\subsection{User Interface}
talk about what tech is used for the ui
\subsection{Backend Infrastructure}
talk about what tech is used for the backend
\subsection{Database}
talk about what tech is used for the db


\section{Software Design}
the model-view-viewmodel follows the seperation of concerns principles blablabla
The MVVM pattern promotes loose coupling and separation of concerns, making the application more maintainable and testable. The View and ViewModel are often connected using data binding techniques, allowing automatic synchronization of data between the two.
By following MVVM, developers can achieve a clear separation of responsibilities, allowing for easier development, testing, and maintenance of the application.
\subsection{Model}
The Model represents the data and business logic of the application. It encapsulates the data structures, data access, validation, and other operations related to the application's data. The Model does not have direct knowledge of the View or ViewModel.
\subsection{View}
The View is responsible for displaying the user interface and capturing user input. It binds to the ViewModel to receive data and update the UI accordingly. The View does not contain any business logic and should be kept as lightweight as possible.
\subsection{ViewModel}
The ViewModel acts as an intermediary between the View and the Model. It provides data and behavior specific to the View's needs. The ViewModel exposes data properties and commands that the View can bind to. It also handles user interactions and updates the Model accordingly. The ViewModel does not have direct knowledge of the View's implementation details.

\section{Features}
