\subsection{Model}
The \emph{model} represents the business logic to solve the problems listed as the requirements of the systems. Once the use cases outlined in \autoref{chap:requirements} are analyzed, the functions of each system components can be determined and then a sequence of operations can be created for each specific use case.
To gain a better overview and understanding of the entities inside the system, a diagram is created.
\begin{figure}[H]
    \centering
    \includegraphics[width=0.8\textwidth]{diagrams/ham-entity.drawio.png}
    \caption{Entity class diagram}
    \label{fig:entity_diagram}
\end{figure}
This diagram illustrates the main entities and its relationship with each other inside the system. As described by the diagram, the main entities in the system consist of \emph{user}, \emph{exercise}, \emph{heartrate}. 
Each \emph{user} within the system has one to many relationship with both \emph{heartrate} and \emph{exercise} entities. These relationships represent the connection between a \emph{user} and their corresponding \emph{heartrate} and \emph{exercise} records. 
Once the entities have been defined, the system's functionalities can now be discussed based on the identified use cases.

\subsubsection{Heart Rate Monitor}
A system sequence can now be defined as a result of analyzing the following use case:
\begin{quotation}
    \enquote{As a health-oriented user, I want the application to show my live heart rate so that I can monitor my cardiovascular activity.} 
\end{quotation}

\begin{figure}[H]
    \centering
    \includegraphics[width=0.9\textwidth]{diagrams/connection-service-onStart.drawio.png}
    \caption{Connection service sequence diagram}
    \label{fig:connection_diagram}
\end{figure}
Since one of the requirements is to monitor real-time heart rate, it is mandatory to establish connection to the heart rate sensor and handle the broadcasted heart rate data. The user should have a way to easily connect and disconnect from the device as desired.
The following sequence will be executed as a way to maintain connection to the BLE heart rate sensor and retrieve the heart rate data.
\begin{enumerate}
    \item A request to start the \emph{connection service} is received.
    \item The service initiates a bluetooth low energy connection to the heart rate sensor device.
    \item After the connection has been established, the \emph{connection service} listens to the heart rate data broadcasted by the heart rate sensor device.
    \item The heart rate sensor broadcasts heart rate data.
    \item The service publishes an event containing the received heart rate data.
    \item A request to stop the \emph{connection service} is received.
    \item The service stops the connection with the heart rate sensor device.
\end{enumerate}

\subsubsection{Activity Monitor}
Based on the analysis of the following use case, a system sequence that outlines the sequence of actions and interactions within the system can now be established.
\begin{quotation}
    \enquote{As an physically active individuals, I want the application to track and display my current activity so that I can keep track of my progress and make adjustments to my physical activities.} 
\end{quotation}

\begin{figure}[H]
    \centering
    \includegraphics[width=0.8\textwidth]{diagrams/activity-monitor-seq.drawio.png}
    \caption{Activity service sequence diagram}
    \label{fig:activity_diagram}
\end{figure}

The following sequence illustrates the steps in monitoring the user's activity, which is one of the key use cases in the system:
\begin{enumerate}
    \item The \emph{activity service} actively listens to the heart rate event broadcasted by the \emph{connection service}.
    \item On heart rate event received, the \emph{activity service} classifies the user's current activity based on the user's current heart rate by using the formula mentioned in \autoref{chap:activity_intensity}
    \item Once the heart rate is classified, the service publishes an event containing the current activity.
\end{enumerate}

\subsubsection{Calculate Energy Expenditure}
Given that one of the requirements defined in the use cases of the system is to calculate burned calories or energy expenditure, it is necessary to implement a feature to that enables the calculation of energy expenditure. 
Based on the formula mentioned in \autoref{chap:energy_expenditure}, it is required to specify the duration during which the calculation will be performed. To facilitate this, it is recommended to create a record of exercise. The exercise should holds the attributes needed for the calculation of the energy expenditure.



