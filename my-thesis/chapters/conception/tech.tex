\section{Technologies}
This section provides an overview of the key technologies required to develop the application. It is important to note that different technologies have their own advantages and disadvantages. Therefore, the goal of this phase is not to find the best technology, but rather to find the most suitable technologies based on the goal of this project and preferences of the writer. 
Android with Kotlin is chosen as the development platform for this project due to its widespread popularity and compatibility with the Model-View-ViewModel architectural pattern. Additionally, Android offers a wide range of development tools and resources, making it suitable for this project.

\subsection{User Interface}
As the main development platform for this project is Android Kotlin, the user interface is designed and implemented using a combination of Kotlin and XML markup with the help of Material Design\footnote{\emph{Material Design} is a design language developed by Google. Homepage: \url{https://m2.material.io/develop/android}}.
Implementing a graph to display live heart rate data can be advantageous to the user. Therefore for smooth and appealing visualization, MPAndroidChart\footnote{\emph{MPAndroidChart} is a chart library for Android, designed for generating interactive and dynamic charts. Github repository: \url{https://github.com/PhilJay/MPAndroidChart}} is used to help with the graph implementation.
\subsection{Backend Infrastructure}
talk about what tech is used for the backend

\subsection{Database}
talk about what tech is used for the db

\subsection{BLE Sensor Device}
talk about what tech is used for the db