\section{Technologies}
\label{chap:tech}
This section provides an overview of the key technologies required to develop the application. It is important to note that different technologies have their own advantages and disadvantages. Therefore, the goal of this phase is not to find the best technology, but rather to find the most suitable technologies based on the goal of this project and preferences of the writer. 
Android with Kotlin is chosen as the development platform for this project due to its widespread popularity and compatibility with the Model-View-ViewModel architectural pattern. Additionally, Android offers a wide range of development tools and resources, making it suitable for this project.

\subsection{User Interface}
As the main development platform for this project is Android Kotlin, the user interface is designed and implemented using a combination of Kotlin and XML markup with the help of Material Design\footnote{\emph{Material Design} is a design language developed by Google. Homepage: \url{https://m2.material.io/develop/android}}.
Implementing a graph to display live heart rate data can be advantageous to the user. Therefore, for smooth and appealing visualization, MPAndroidChart\footnote{\emph{MPAndroidChart} is a chart library for Android, designed for generating interactive and dynamic charts. Github repository: \url{https://github.com/PhilJay/MPAndroidChart}} is used to help with the graph implementation.
\subsection{Backend Infrastructure}
As mentioned before, the backend of the application is developed using Kotlin due to its compatibility with the MVVM architecture. 
Furthermore, Kotlin provides pre-defined classes such as the \emph{Service}\footnote{\emph{Service} is an application component that can perform long-running operations in the background. URL: \url{https://developer.android.com/guide/components/services}} class, which enables smooth execution of long running task. In the scope of this project, \emph{Service} class is especially useful to implement the \emph{connection service} and \emph{exercise service}.

\subsection{Database}
A database is required to store and manage data related to the \emph{model}. In the context of this project, ROOM\footnote{\emph{ROOM} is a persistence library that provides an abstraction layer over SQLite to allow for more robust database access while harnessing the full power of SQLite. URL: \url{https://developer.android.com/jetpack/androidx/releases/room}} database, which is built on top of SQLite\footnote{\emph{SQLite} is a C-language library that implements a small, fast, self-contained, high-reliability, full-featured, SQL database engine. Homepage: \url{https://www.sqlite.org/index.html}} has been selected as the application's database.

\subsection{BLE Sensor Device}
Once all the technologies required to develop the application have been decided, the next step is to select a suitable heart-rate sensor device. Considering factors like accuracy, performance, and affordability, Polar H9\footnote{\emph{Polar H9} is a heart rate sensor made by Polar. Homepage: \url{https://www.polar.com/en/sensors/h9-heart-rate-sensor}} is chosen as the preferred device due to its reputation for providing reliable heart-rate measurements and long battery life.