\section{Technologies}
\label{chap:tech}
This section provides an overview of the key technologies required to develop the application. 
The goal of this phase is to find the most suitable technologies based on the goal of this project and preferences of the writer. 
There are two major platforms to consider for this development: Android and iOS.

Several factors advocate for the choice of Android. 
Firstly, Android has a larger user base \autocite{statcounter_os_market_share}, which implies the application would be accessible to a greater number of potential users. 
Secondly, Android devices tend to be more affordable, which means the application could reach users from different economic backgrounds.
Additionally, Android offers a wide range of development tools and resources that could support the development process.
Lastly, the fact that the writer already owns an Android device simplifies the testing and debugging process, making Android a sensible choice for the development platform of the application.

Once the platform has been determined, the next step is to select the programming language to build the application.
Android supports two major languages: Java and Kotlin. Given the real-time heart rate monitoring involved in this project, Kotlin, with its structured concurrency, becomes more favored. The capability to run multiple threads in parallel is ideal for real-time monitoring tasks.
Moreover, Google supports Kotlin as the first-class language for Android Applications. This means that Kotlin has strong tooling support in Android Studio\footnote{\emph{Android Studio} is Google's official IDE for Android}. \autocite{kotlin_android}
These factors make Kotlin a practical choice for developing the application.

\subsection{User Interface}
As the main development platform for this project is Android Kotlin, the user interface is designed and implemented using a combination of Kotlin and XML with the help of Material Design\footnote{\emph{Material Design} is a design language developed by Google. Homepage: \url{https://m2.material.io/develop/android}}.
Implementing a graph to display live heart rate data can be advantageous to the user. Therefore, for smooth and appealing visualization, MPAndroidChart\footnote{\emph{MPAndroidChart} is a chart library for Android, designed for generating interactive and dynamic charts. Github repository: \url{https://github.com/PhilJay/MPAndroidChart}} is used to help with the graph implementation due to its wide collection of charts. Additionally, it also supports dynamic graph, allowing real-time data visualization.
\subsection{Backend Infrastructure}
The backend of the application is developed using Kotlin with Android Jetpack\footnote{\emph{Jetpack} is a suite of libraries, tools, and guidance to help developers write apps easier. URL: \url{https://developer.android.com/jetpack}} due to its compatibility with the MVVM architecture. 
Furthermore, Kotlin provides pre-defined classes such as the \emph{Service}\footnote{Based on The Official Android Documentation, \emph{Service} is an application component that can perform long-running operations in the background.\autocite{android-services}} class, which enables smooth execution of long running task. In the scope of this project, \emph{Service} class is especially useful to implement the \texttt{ConnectionService} and \texttt{ExerciseService}.

\subsection{Database}
A database is required to store and manage data. In the context of this project, ROOM\footnote{\emph{ROOM} is a persistence library that provides an abstraction layer over SQLite to allow for more robust database access while harnessing the full power of SQLite. URL: \url{https://developer.android.com/jetpack/androidx/releases/room}} database, which is built on top of SQLite\footnote{\emph{SQLite} is a C-language library that implements a small, fast, self-contained, high-reliability, full-featured, SQL database engine. Homepage: \url{https://www.sqlite.org/index.html}} has been selected as the application's database due to its official recommendation and integration with Android development.

\subsection{BLE Sensor Device}
Once all the technologies required to develop the application have been decided, the next step is to select a suitable heart-rate sensor device. Considering factors like accuracy, performance, and affordability, Polar H9\footnote{\emph{Polar H9} is a heart rate sensor made by Polar. Homepage: \url{https://www.polar.com/en/sensors/h9-heart-rate-sensor}} is chosen as the preferred device due to its affordability and reputation for providing reliable heart-rate measurements and long battery life. Additionally, Polar provides a software development kit (SDK)\footnote{The SDK will be discussed in \autoref{chap:polarblesdk}} that simplifies the process of developing applications for polar heart rate sensors.