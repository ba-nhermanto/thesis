\chapter{Introduction}

Health monitoring has become more and more important at the moment, with people seeking ways to track and maintain their well-being. 
One of the most important aspect of health monitoring is the heart rate measurement as it can provide valuable insights into one's fitness level and overall wellness.

Wearable technology, such as smartwatches and fitness trackers, has become popular for monitoring physical activity, heart rate, and fitness progress.
However, these devices often come with limited functionality or even more than needed functionality and according to a research conducted in the US National Institute of Health \autocite{holko2022wearable}, 49,4\% of adults at federally qualified health centres find those devices are too expensive. 
Furthermore, over 20\% of the people who responded to the survey do not know how to use them but would like to have a fitness tracker and over 15\% do not know how it can help them but would like to learn about it. 
The data suggests that many individuals express interest in wearable devices but struggle with using them. 
This may be due to complex user interfaces and to many people, smartwatches can be confusing because of their advanced features that may overshadow the primary ones. 
In addition to that, there are several privacy issues in wearable technology mentioned by Kapoor et al. in \autocite{kapoor2020privacy}, some of them are: data breaches on cloud servers or wearable device and lack of user control over their health

\section{Background and Motivation}
The motivation behind this project is to design and implement a health and activity monitor application using a wearable Bluetooth Low Energy (BLE) sensor that provides real-time monitoring of health and fitness data. 
The application aims to provide a convenient, practical, and affordable solution to monitor health and fitness, which can help individuals achieve their health goals and make informed decisions about their well-being.
The application should provide users control over their health data. For instance, users should have the ability to manipulate their health data such as deleting their health data. The application should store collected data in local database instead of relying on cloud servers or the wearable device to minimize the risk of data breaches.
As the proposed application focuses solely on monitoring users' heart rate, it is essential for it to be straightforward and functional, without any unnecessary features. 
The practical usage should be simple, with users only required to wear the sensor and connect it to the application via bluetooth low energy.


\section{Goal and Scope}
The goal of this project is to build a functional prototype of a mobile application designed to monitor users' heart rate and activity.
The application should be able to establish connection with a specific heart rate sensor device and retrieve the heart rate data.
The retrieved data will then be stored in a database and can be accessed later for visualization purposes.
The application should also be capable of determining users' activity based on their heart rate and age.
Furthermore, the application should also be able to calculate users' energy expenditure during their exercise based on their age, gender, weight, duration of exercise.
The prototype will also feature an intuitive user interface to present this information to the user in an easily understandable format.

The result of this project will not necessarily replace the current applications in the market, rather it is an exploratory work, and may be used as a base for future projects.
Overall, the application has the potential to improve the health and wellbeing of individuals by providing real-time data on their heart rate, monitoring their physical activity and calculating calories burnt during their exercise. Additionally, the application will also give users control over their data.
