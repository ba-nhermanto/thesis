
%----------------------------------------------------------------------------------------
%   Bachelor Thesis: Nathanael I Hermanto 570619
% 	Template: Abschlussarbeiten des Studiengangs Angewandte Informatik an der HTW Berlin
% 	C. Schmidt 
%----------------------------------------------------------------------------------------
%	Pakete und Konfigurationen
%----------------------------------------------------------------------------------------
\documentclass[oneside,bibliography=totocnumbered,BCOR=5mm]{scrbook}% Voreinstellungen entfernt.

\usepackage[latin1]{inputenc}
\usepackage{amsmath, amsthm, amssymb}
\usepackage[ngerman, main=english]{babel} % Language hyphenation and typographical rules
\usepackage{marvosym}
\usepackage{graphics}
\usepackage{csquotes}
\newtheorem{satz}{Satz}[chapter]
\theoremstyle{definition} 
\newtheorem{definition}[satz]{Definition} 
\theoremstyle{definition} 
\newtheorem{lemma}[satz]{Lemma} 
\theoremstyle{definition} 
\newtheorem{bemerkung}[satz]{Bemerkung}
\theoremstyle{definition} 
\newtheorem{korollar}[satz]{Korollar} 
\theoremstyle{definition}
\newtheorem{beispiel}[satz]{Beispiel} 
\theoremstyle{definition} 
\newtheorem{algorithmus}{Algorithmus} 
\newenvironment{beweis}{\begin{proof}[Beweis]}{\end{proof}}
\usepackage[hyphens]{url}
\usepackage{hyperref}

%----------------------------------------------------------------------------------------
%	BIB.-Datei und Quellenverwaltung
%----------------------------------------------------------------------------------------
\usepackage[backend=bibtex, style=numeric]{biblatex}
\addbibresource{template.bib}
%\usepackage{natbib} % use natbib for references 
%----------------------------------------------------------------------------------------
\usepackage{blindtext} % Package to generate dummy text throughout this template 

\usepackage[sc]{mathpazo} % Use the Palatino font
\usepackage[T1]{fontenc} % Use 8-bit encoding that has 256 glyphs
\linespread{1.05} % Line spacing - Palatino needs more space between lines
\usepackage{microtype} % Slightly tweak font spacing for aesthetics

\usepackage[hmarginratio=1:1,top=32mm,columnsep=20pt]{geometry} % Document margins
\usepackage[hang, small,labelfont=bf,up,textfont=it,up]{caption} % Custom captions under/above floats in tables or figures
\usepackage{booktabs} % Horizontal rules in tables
\usepackage{lettrine} % The lettrine is the first enlarged letter at the beginning of the text
\usepackage{enumitem} % Customized lists
\setlist[itemize]{noitemsep} % Make itemize lists more compact

%\usepackage{abstract} % Allows abstract customization
%\renewcommand{\abstractnamefont}{\normalfont\bfseries} % Set the "Abstract" text to bold
%\renewcommand{\abstracttextfont}{\normalfont\small\itshape} % Set the abstract itself to small italic text

%\usepackage{fancyhdr} % Headers and footers
%\pagestyle{fancy} % All pages have headers and footers
%\fancyhead{} % Blank out the default header
%\fancyfoot{} % Blank out the default footer
%\fancyhead[C]{Ethics in Progress (EiP) $\bullet$ 2019 } % Custom header text
%\fancyfoot[RO,LE]{\thepage} % Custom footer text

\usepackage{titling} % Customizing the title section

%----------------------------------------------------------------------------------------
%	Listings
%----------------------------------------------------------------------------------------
\usepackage{listings}
\usepackage{color}

\definecolor{mygreen}{rgb}{0,0.6,0}
\definecolor{mygray}{rgb}{0.5,0.5,0.5}
\definecolor{mymauve}{rgb}{0.58,0,0.82}

\lstset{ 
  backgroundcolor=\color{white},   % choose the background color; you must add \usepackage{color} or \usepackage{xcolor}; should come as last argument
  basicstyle=\footnotesize,        % the size of the fonts that are used for the code
  breakatwhitespace=false,         % sets if automatic breaks should only happen at whitespace
  breaklines=true,                 % sets automatic line breaking
  captionpos=b,                    % sets the caption-position to bottom
  commentstyle=\color{mygreen},    % comment style
  deletekeywords={...},            % if you want to delete keywords from the given language
  escapeinside={\%*}{*)},          % if you want to add LaTeX within your code
  extendedchars=true,              % lets you use non-ASCII characters; for 8-bits encodings only, does not work with UTF-8
  firstnumber=1,                % start line enumeration with line 1000
  frame=single,	                   % adds a frame around the code
  keepspaces=true,                 % keeps spaces in text, useful for keeping indentation of code (possibly needs columns=flexible)
  keywordstyle=\color{blue},       % keyword style
  language=Octave,                 % the language of the code
  morekeywords={*,...},            % if you want to add more keywords to the set
  numbers=left,                    % where to put the line-numbers; possible values are (none, left, right)
  numbersep=5pt,                   % how far the line-numbers are from the code
  numberstyle=\tiny\color{mygray}, % the style that is used for the line-numbers
  rulecolor=\color{black},         % if not set, the frame-color may be changed on line-breaks within not-black text (e.g. comments (green here))
  showspaces=false,                % show spaces everywhere adding particular underscores; it overrides 'showstringspaces'
  showstringspaces=false,          % underline spaces within strings only
  showtabs=false,                  % show tabs within strings adding particular underscores
  stepnumber=1,                    % the step between two line-numbers. If it's 1, each line will be numbered
  stringstyle=\color{mymauve},     % string literal style
  tabsize=2,	                   % sets default tabsize to 2 spaces
  title=\lstname                   % show the filename of files included with \lstinputlisting; also try caption instead of title
}

%----------------------------------------------------------------------------------------
%	Haupttextteil
%----------------------------------------------------------------------------------------

\begin{document}

% Titelseite
% \pagestyle{empty}       % keine Seitennummer
\begin{titlepage}
\begin{center}
\includegraphics{HTW_Berlin_Logo_farbig.jpg}
\linebreak[4]
\linebreak[4]
\linebreak[4]
\linebreak[4]
\textit{\large Titel der Abschlussarbeit}
\linebreak[4]
\linebreak[4]
\linebreak[4]
Abschlussarbeit 
\linebreak[4]
\linebreak[4]
zur Erlangung des akademischen Grades: 
\linebreak[4]
\linebreak[4]
\textbf{Bachelor of Science (B.Sc.)} oder \textbf{Master of Science (M.Sc.)}
\linebreak[4]
\linebreak[4]
an der
\linebreak[4]
\linebreak[4]
Hochschule f\"ur Technik und Wirtschaft (HTW) Berlin
\linebreak[4]
Fachbereich 4: Informatik, Kommunikation und Wirtschaft
\linebreak[4]
Studiengang \textit{Angewandte Informatik}
\linebreak[4]
\linebreak[4]
\linebreak[4]
1. Gutachter\_in: Titel akademischer Grad Vorname Nachname\linebreak[4]
2. Gutachter\_in: Titel akademischer Grad Vorname Nachname\linebreak[4]
\linebreak[4]
\linebreak[4]
\linebreak[4]
\linebreak[4]
Eingereicht von Vorname Nachname [Matrikelnr.]
\linebreak[4]
\linebreak[4]
\linebreak[4]
\linebreak[4]
Datum

\end{center}
\end{titlepage}
\newpage    % Seitenwechsel

\thispagestyle{empty}       % keine Seitennummer
% vertikaler Leerraum
\vspace*{2.2cm}
\noindent %kein Einzug
{\Huge Danksagung}\\
\vspace*{1.6cm} \\

% Kopfzeilen (automatisch erzeugt)
%\pagestyle{headings}
[Text der Danksagung]

% Seite mit Abstracts
\newpage
\thispagestyle{empty}       % keine Seitennummer
\section*{Zusammenfassung}
[Text der Zusammenfassung]

\section*{Abstract}
[Summary of the thesis]


\clearpage
%Seite 1
\pagenumbering{roman}% Seitennummerierung "roemisch"
%\setcounter{page}{1} 

\tableofcontents  


%Seite 

 \listoffigures
 
 %Seite 6

 \listoftables
 


 \lstlistoflistings

.


\newpage

\pagenumbering{arabic} 
%----------------------------------------------------------------------------------------
%	CHAPTERS
%----------------------------------------------------------------------------------------
\chapter{Introduction}

This is the introduction chapter 

\section{Background and Motivation}
Background and Motivation section

\section{Goal and Scope}
Goal and Scope section



%----------------------------------------------------------------------------------------
%	REFERENCE LIST
%----------------------------------------------------------------------------------------


% \bibliographystyle{apalike}
% \bibliographystyle{ksfh_nat} % ein anderer Stil
% \bibliography{science} 

%\begin{thebibliography}{XX}
%\bibitem[Bielecki, Rutkowski(2002)]{bielecki02}T. Bielecki, M. Rutkowski, Credit Risk: Modeling, Valuation and Hedging, Springer (2002).
%\bibitem[Jarrow, Turnbull(1995)]{jarrow95} R.A. Jarrow, S. Turnbull, Pricing derivatives on financial securities subject to credit risk, Journal of Finance, 50:1 (1995) 53--85.
%\bibitem[Marshall, Olkin(1967)]{marshall67} A.W. Marshall, I. Olkin, A multivariate exponential distribution, Journal of the American Statistical Association, 62 (1967), pp. 30--44.
%\bibitem[Sch\"onbucher(2003)]{schoenbucher03} P.J. Sch\"onbucher, Credit Derivatives Pricing Models, Wiley (2003).
%\end{thebibliography}

\printbibliography[
heading=bibintoc,
title={Quellenverzeichnis}
]


\newpage

\chapter{Abk\"urzungsverzeichnis}
\newpage
\chapter{Glossar}




\begin{appendix}

\pagenumbering{Roman}

\chapter{Appendix}


\section{Quell-Code}

\section{Tipps zum Schreiben Ihrer Abschlussarbeit}

\begin{itemize}
\item Achten Sie auf eine neutrale, fachliche Sprache. Keine \glqq{}Ich\grqq{}-Form.
\item Zitieren Sie zitierf\"ahige und -w\"urdige Quellen (z.B. wissenschaftliche Artikel und Fachb\"ucher; nach M\"oglichkeit keine Blogs und keinesfalls Wikipedia\footnote{Wikipedia selbst empfiehlt, von der Zitation von Wikipedia-Inhalten im akademischen Umfeld Abstand zu nehmen \autocite{wikipedia2019}.}). 
\item Zitieren Sie korrekt und homogen.
\item Verwenden Sie keine Fu{\ss}noten f\"ur die Literaturangaben.
\item Recherchieren Sie ausf\"uhrlich den Stand der Wissenschaft und Technik.
\item Achten Sie auf die Qualit\"at der Ausarbeitung (z.B. auf Rechtschreibung).
\item Informieren Sie sich ggf. vorab dar\"uber, wie man wissenschaftlich arbeitet bzw. schreibt:
\begin{itemize}
\item Mittels Fachliteratur\footnote{Z.B. \autocite{balzert2011}, \autocite{franck2013}}, oder
\item Beim Lernzentrum\footnote{Weitere Informationen zum Schreibcoaching finden sich hier: \url{https://www.htw-berlin.de/studium/lernzentrum/studierende/schreibcoaching/}; letzter Zugriff: 13 VI 19.}.
\end{itemize}
\item Nutzen Sie \LaTeX\footnote{Kein Support bei Installation, Nutzung und Anpassung allf\"alliger \LaTeX-Templates!}.
\end{itemize}



\newpage
% Letzte Seite
\thispagestyle{empty}       % keine Seitennummer
%\vspace*{18cm}
\noindent

\section*{Eidesstattliche Versicherung}
Hiermit versichere ich an Eides statt durch meine Unterschrift, dass ich die vorstehende Arbeit selbstst\"andig und ohne fremde Hilfe angefertigt und alle Stellen, die ich w\"ortlich oder ann\"ahernd w\"ortlich aus Ver\"offentlichungen entnommen habe, als solche kenntlich gemacht habe, mich auch keiner anderen als der angegebenen Literatur oder sonstiger Hilfsmittel bedient habe. Die Arbeit hat in dieser oder \"ahnlicher Form noch keiner anderen Pr\"ufungsbeh\"orde vorgelegen.\\
\linebreak[4]
\linebreak[4]
\linebreak[4]
\linebreak[4]
-------------------------------------------------------\linebreak[4]
Datum, Ort, Unterschrift


\end{appendix}


\end{document}

